% !TeX document-id = {2870843d-1baa-4f6a-bd0a-a5c796104a32}
% !BIB TS-program = biber
% !TeX encoding = UTF-8
% TU Delft beamer template

\documentclass[aspectratio=169]{beamer}
\usepackage[english]{babel}
\usepackage{csquotes}
\usepackage{calc}
\usepackage[absolute,overlay]{textpos}
\usepackage{graphicx}
\usepackage{subfig}
\usepackage{mathtools}
\usepackage{multicol}
\usepackage{amsfonts}
\usepackage{amsthm}
\usepackage{comment}
\usepackage{siunitx}
\usepackage{MnSymbol,wasysym}
\usepackage{array}
\usepackage{qrcode}
\usepackage{tikz}

\setbeamertemplate{navigation symbols}{} % remove navigation symbols
\mode<presentation>{\usetheme[verticalbar=false]{tud}}
\setbeamercovered{transparent}

% BIB SETTINGS
\usepackage[
backend=biber,
giveninits=true,
maxnames=30,
maxcitenames=20,
uniquename=init,
url=false,
style=authoryear,
]{biblatex}
\addbibresource{bibfile.bib}
\setlength\bibitemsep{0.3cm} % space between entries in the reference list
\renewcommand{\bibfont}{\normalfont\scriptsize}
\setbeamerfont{footnote}{size=\tiny}
\renewcommand{\cite}[1]{\footnote<.->[frame]{\fullcite{#1}}}
\setlength{\TPHorizModule}{\paperwidth}
\setlength{\TPVertModule}{\paperheight}

\newcommand{\absimage}[4][0.5,0.5]{%
	\begin{textblock}{#3}%width
		[#1]% alignment anchor within image (centered by default)
		(#2)% position on the page (origin is top left)
		\includegraphics[width=#3\paperwidth]{#4}%
\end{textblock}}

\newcommand{\mininomen}[2][1]{{\let\thefootnote\relax%
		\footnotetext{\begin{tabular}{*{#1}{@{\!}>{\centering\arraybackslash}p{1em}@{\;}p{\textwidth/#1-2em}}}%
				#2\end{tabular}}}}


% to solve spurious empty pages before a frame containing footnotes
\setbeamertemplate{footnote}{%
	\makebox[1em][l]{\insertfootnotemark}%
	\begin{minipage}{\dimexpr\linewidth-1em}
		\usebeamerfont{footnote}\insertfootnotetext
	\end{minipage}\vskip 0pt}


\title[]{Improving Data Analysis of Incomplete Tabular Data by Learning Representation of Data via Graph Neural Networks}
\institute[]{Faculty of ICT, Mahidol University}
\author{Phaphontee Yamchote}
\date{November 18, 2023}

\AtBeginSection[]
{
	\begin{frame}[noframenumbering,plain]
		\frametitle{Outline: Next Topic}
		\tableofcontents[currentsection]
%		\addtocounter{framenumber}{-1}
	\end{frame}
}
\theoremstyle{definition}
\newtheorem{prop}[theorem]{Proposition}
\newtheorem{lem}[theorem]{Lemma}
\newtheorem{eq}[theorem]{Equation}

\newcommand{\R}{\mathbb{R}}
\newcommand{\N}{\mathbb{N}}
\newcommand{\Z}{\mathbb{Z}}
\newcommand{\Q}{\mathbb{Q}}

\begin{document}
	
	\addtocounter{framenumber}{-1}
	{
		\setbeamertemplate{footline}{\usebeamertemplate*{minimal footline}}
		\frame{\titlepage}
	}
	
	\begin{frame}[noframenumbering,plain]{Outline}
		\tableofcontents
	\end{frame}
	
	\section{Introduction}
	

	\begin{frame}[t,noframenumbering,plain]{\bibname}
	\tiny \begin{thebibliography}{99} % Bibliography - this is intentionally simple in this template
			\bibitem{Borisov} Borisov, Vadim, et al. "Deep neural networks and tabular data: A survey." arXiv preprint arXiv:2110.01889 (2021).
			\bibitem{PET} Du, Kounianhua, et al. "Learning enhanced representations for tabular data via neighborhood propagation." arXiv preprint arXiv:2206.06587 (2022).
			%	
			\bibitem{Gorishniy} Gorishniy, Yury, et al. "Revisiting deep learning models for tabular data." Advances in Neural Information Processing Systems 34 (2021): 18932-18943.
			%
			\bibitem{Grinsztajn} Grinsztajn, Léo, Edouard Oyallon, and Gaël Varoquaux. "Why do tree-based models still outperform deep learning on tabular data?." arXiv preprint arXiv:2207.08815 (2022).

			\bibitem{criteo} Jean-Baptiste Tien, joycenv, Olivier Chapelle. (2014). Display Advertising Challenge. Kaggle. https://kaggle.com/competitions/criteo-display-ad-challenge
			%
			\bibitem{lightgbm} Ke, Guolin, et al. "Lightgbm: A highly efficient gradient boosting decision tree." Advances in neural information processing systems 30 (2017).
			%
			\bibitem{GCN} Kipf, Thomas N., and Max Welling. "Semi-supervised classification with graph convolutional networks." arXiv preprint arXiv:1609.02907 (2016).
			%	
			\bibitem{fi-gnn} Li, Zekun, et al. "Fi-gnn: Modeling feature interactions via graph neural networks for ctr prediction." Proceedings of the 28th ACM International Conference on Information and Knowledge Management. 2019.
			%
			\bibitem{PyG} “PyG Documentation - Pytorch\_Geometric Documentation.” PyG Documentation - Pytorch\_Geometric Documentation, pytorch-geometric.readthedocs.io. Accessed 14 Nov. 2022. <https://pytorch-geometric.readthedocs.io>
			%
			\bibitem{Rendle} Rendle, Steffen. "Factorization machines." 2010 IEEE International conference on data mining. IEEE, 2010.
			%
			\bibitem{transformerconv} Shi, Yunsheng, et al. "Masked label prediction: Unified message passing model for semi-supervised classification." arXiv preprint arXiv:2009.03509 (2020).
			%
			\bibitem{Shwartz} Shwartz-Ziv, Ravid, and Amitai Armon. "Tabular data: Deep learning is not all you need." Information Fusion 81 (2022): 84-90.
			%	
			\bibitem{avazu} Steve Wang, Will Cukierski. (2014). Click-Through Rate Prediction. Kaggle. https://kaggle.com/competitions/avazu-ctr-prediction
		\end{thebibliography}
	\end{frame}
	
	\begin{frame}[noframenumbering,plain]
		\begin{center}
			{\Large Q\&A}
		\end{center}
	\end{frame}
	
%	\section*{More on Main Theorem}
%	
%	\begin{frame}[t,noframenumbering,plain]{\secname}
%		\begin{prop}
%			Let $C_1$ and $C_2$ be cluster graphs in $\mathcal{G}(n)$ for $n \in \N$. Then
%			$[C_1]_\sim = [C_2]_\sim$ if and only if $C_1 = C_2$.
%		\end{prop}
%	\end{frame}
%	\begin{frame}[t,noframenumbering,plain]{\secname}
%		\begin{lem} \label{lem1}
%			The set of partition of the set $N_n$, named $P(n)$, one-to-one corresponds to $\mathfrak{G}(n)$
%		\end{lem}
%		\begin{lem} \label{lem2}
%			$\mathcal{DM}$ one-to-one corresponds to $P(n)$.
%		\end{lem}
%	\end{frame}
	
	
\end{document}

